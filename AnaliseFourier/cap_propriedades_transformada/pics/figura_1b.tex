
 \documentclass{standalone}
 \usepackage{pstricks-add}
 \usepackage{pstricks,pst-plot}
 \usepackage[dvips]{graphicx}
 \usepackage{pst-math}
 \usepackage{pst-plot}
 \usepackage{pst-circ}
 \usepackage[brazil]{babel}
 \usepackage[utf8]{inputenc}
 \usepackage[T1]{fontenc}
 \usepackage{amsmath}
 \usepackage{amssymb}
 \usepackage{amsthm}
 \usepackage{mathtools}
 \newcommand{\sen}{\operatorname{sen}\,}
 \newcommand{\senh}{\operatorname{senh}\,}
 \renewcommand{\sin}{\operatorname{sen}\,}
 \renewcommand{\sinh}{\operatorname{senh}\,}
 \begin{document}\psset{unit =1cm, linewidth=1\pslinewidth}
 \begin{pspicture}(-5.3,-.8)(5.5,3.1)
 \psaxes{->}(0,0)(-5.0,-.1)(5.3,2.6)
\psset{linecolor=blue}
\psplot[plotstyle=curve,plotpoints=200]{-5}{-3}{0}
\psplot[plotstyle=curve,plotpoints=200]{-3}{-2}{x 3 add x mul -1 mul}
\psplot[plotstyle=curve,plotpoints=200]{-2}{-1}{x -1 mul}
\psplot[plotstyle=curve,plotpoints=200]{-1}{0}{x -1 mul x mul -1 mul}
\psplot[plotstyle=curve,plotpoints=200]{0}{1}{x x mul}
\psplot[plotstyle=curve,plotpoints=200]{1}{2}{x}
\psplot[plotstyle=curve,plotpoints=200]{2}{3}{3 x sub  x mul}
\psplot[plotstyle=curve,plotpoints=200]{3}{5.3}{0}
\psline[linestyle=dotted](-2,0)(-2,2)
\psline[linestyle=dotted](-1,0)(-1,1)
\psline[linestyle=dotted](2,0)(2,2)
\psline[linestyle=dotted](1,0)(1,1)

\rput(0,2.8){$|\mathcal{F}\left\{f'(t)\right\}|$}
\rput(5.2,.3){$w$}
\end{pspicture}
\end{document}
