%Este trabalho está licenciado sob a Licença Creative Commons Atribuição-CompartilhaIgual 3.0 Não Adaptada. Para ver uma cópia desta licença, visite https://creativecommons.org/licenses/by-sa/3.0/ ou envie uma carta para Creative Commons, PO Box 1866, Mountain View, CA 94042, USA.

\chapter{Propriedades das Séries de Fourier}

\section{Teorema de Parseval}

\begin{defn} 
Define-se a potência média de um função periódica $f(t)$ como
$$\overline{P}_f=\frac{1}{T}\int_0^T |f(t)|^2dt$$
\end{defn}
\begin{ex}{\label{ex_1_cap_5}} A potência média da função $f(t)=A\cos(wt)$ é dada por
\begin{eqnarray*}
\overline{P}_f&=&\frac{1}{T}\int_0^T |f(t)|^2dt\\
&=&\frac{1}{T}\int_0^{T} A^2\cos\left(\frac{2\pi}{T} t\right)^2dt\\
&=&\frac{A^2}{T}\int_0^{T} \left(\frac{\cos\left(\frac{4\pi}{T} t\right)+1}{2}  \right)dt\\
&=&\frac{A^2}{2}
\end{eqnarray*}
 onde se usou que $w=\frac{2\pi}{T}$ e identidade trigonométrica dada por:
$$\cos^2(x)=\left(\frac{e^{ix}+e^{-ix}}{2}\right)^2=\frac{e^{2ix}+2+e^{-2ix}}{4}=\frac{\cos(2x)+1}{2}.$$
 \end{ex}
\begin{ex} Seja $V(t)=A\cos(wt)$ uma fonte de tensão com frequência $w=60$\ \!\!Hz $=120\pi$\ \!\!rad/s ligado a um resistor de resitência $R\Omega$. A potência no resistor é
$$
P(t)=\frac{V(t)^2}{R}
$$
e a potência média $P_m$ é
$$
P_m=\frac{1}{T}\int_0^TP(t)dt=\frac{1}{T}\int_0^T\frac{V(t)^2}{R}dt,
$$
onde $T=\frac{1}{60}s$. Por outro lado, a potência média é calculada em termos da tensão média por
$$
P_m=\frac{V_m^2}{R},
$$
ou seja,
\begin{equation}{\label{valor_RMS}}
V_m^2=\frac{1}{T}\int_0^T V(t)^2 dt.
\end{equation}
O exemplo \ref{ex_1_cap_5} nos dá o valor da potência média do sinal $V(t)=A\cos(wt)$. Logo,
$$
V_m=\frac{A}{\sqrt{2}}.
$$
Se $V_m=127V$, então a amplitude do sinal é aproximadamente $A\approx 180$.
\end{ex}
\begin{obs}Na expressão (\ref{valor_RMS}), $V_m$ também é chamado de valor RMS do sinal $v(t)$ (Root mean square):
$$
V_{RMS}=\sqrt{\frac{1}{T}\int_0^T V(t)^2 dt}.
$$
\end{obs}

  

\begin{teo}[Teorema de Parseval] Seja $f(t)$ uma função periódica representável por uma série de Fourier, então vale a seguinte identidade.
\begin{equation}\label{teo_parseval} 
\frac{1}{T}\int_0^T |f(t)|^2dt=\sum_{n=-\infty}^\infty |C_n|^2.
 \end{equation}
\end{teo}
\begin{proof}
\begin{eqnarray*}
 \frac{1}{T}\int_0^T |f(t)|^2dt&=&\frac{1}{T}\int_0^T f(t)\overline{f(t)}dt
\end{eqnarray*}
 Como $\displaystyle f(t)=\sum_{n=-\infty}^\infty C_n e^{iw_n t}$, temos
 \begin{eqnarray*}
  \overline{f(t)}&=&\overline{\sum_{n=-\infty}^\infty C_n e^{iw_n t}}
  =\sum_{n=-\infty}^\infty \overline{C_n}~ \overline{e^{iw_n t}}
  =\sum_{n=-\infty}^\infty \overline{C_n} e^{-iw_n t}
 \end{eqnarray*}
Substituindo esta expressão para $\overline{f(t)}$ na definição de potência média, temos:
\begin{eqnarray*}
 \frac{1}{T}\int_0^T |f(t)|^2dt&=&\frac{1}{T}\int_0^T f(t)\overline{f(t)}dt=\frac{1}{T}\int_0^Tf(t)\left[\sum_{n=-\infty}^\infty \overline{C_n} e^{-iw_n t}\right] dt\\
 &=&\frac{1}{T}\sum_{n=-\infty}^\infty\left[\overline{C_n}\int_0^Tf(t)e^{-iw_nt}dt\right]
 \end{eqnarray*}
 Como $C_n=\frac{1}{T}\int_0^Tf(t)e^{-iw_nt}dt$, temos:
\begin{eqnarray*}
 \frac{1}{T}\int_0^T |f(t)|^2dt&=&\sum_{n=-\infty}^\infty\overline{C_n}C_n = \sum_{n=-\infty}^\infty|C_n|^2
 \end{eqnarray*}
 
\end{proof}


\begin{ex}\label{ex_quadrada_parseval} Seja $g(t)$ um função dada no exemplo \ref{ex_quadrada}, isto é,
\begin{eqnarray*}
g(t)&=&-1, \ \ -1< t<0\\
g(t)&=&0, \ \ t=0\ \hbox{ou}\ t=1\\
g(t)&=&1, \ \ 0< t<1\\
g(t+2)&=&g(t),\ \ \forall t\in\mathbb{R}.
\end{eqnarray*}
\begin{figure}[!ht]
\begin{center}
\psset{unit =2cm, linewidth=1\pslinewidth}
 \begin{pspicture}(-1.3,-1.3)(4.3,1.3)
 \psaxes{->}(0,0)(-1.1,-1.2)(4.2,1.2)
\psplot[plotstyle=curve,linewidth=2\pslinewidth,linecolor=blue]{-1}{0}{-1}
\psplot[plotstyle=curve,linewidth=2\pslinewidth,linecolor=blue]{0}{1}{1}
\psplot[plotstyle=curve,linewidth=2\pslinewidth,linecolor=blue]{1}{2}{-1}
\psplot[plotstyle=curve,linewidth=2\pslinewidth,linecolor=blue]{2}{3}{1}
\psplot[plotstyle=curve,linewidth=2\pslinewidth,linecolor=blue]{3}{4}{-1}
\pscircle[linecolor=blue](0,-1){.05}
\pscircle[linecolor=blue](0,1){.05}
\pscircle[linecolor=blue](1,-1){.05}
\pscircle[linecolor=blue](1,1){.05}
\pscircle[linecolor=blue](2,-1){.05}
\pscircle[linecolor=blue](2,1){.05}
\pscircle[linecolor=blue](3,-1){.05}
\pscircle[linecolor=blue](3,1){.05}
\pscircle[linecolor=blue](-1,-1){.05}
\pscircle[linecolor=blue](4,-1){.05}
\psset{linecolor=blue}
\qdisk(-1,0){.05}
\qdisk(0,0){.05}
\qdisk(1,0){.05}
\qdisk(2,0){.05}
\qdisk(3,0){.05}
\qdisk(4,0){.05}
\rput(.3,1.3){$y=g(t)$}
\rput(4.1,.1){$t$}
\end{pspicture}
\end{center}
\end{figure}
Vimos no exemplo \ref{ex_quadrada} que sua expansão em série de Fourie é da forma:
$$
g(t)=\frac{4}{\pi}\left(\sen(\pi t)+\frac{1}{3}\sen(3\pi t)+\frac{1}{5}\sen(5\pi t)+\cdots\right).
$$
Calcularemos agora a potência média desta função através de sua representação no tempo e depois em frequência:
\begin{eqnarray*}
 \overline{P_f}=\frac{1}{T}\int_0^T |g(t)|^2dt=\frac{1}{2}\int_0^2 |g(t)|^2dt=\frac{1}{2}\int_0^2 1dt=1
\end{eqnarray*}
Alternativamente, temos pelo Teorema de Parseval:
\begin{eqnarray*}
 \overline{P_f}=\sum_{n=-\infty}^\infty |C_n|^2=\sum_{n=-\infty}^\infty \left|\frac{a_n-ib_n}{2}\right|^2=\frac{1}{4}\sum_{n=-\infty}^\infty |b_n|^2
\end{eqnarray*}
Como $b_{-n}=b_n$, temos que $|b_{-n}|=|b_n|$ e ainda temos que $b_0=0$, portanto: 
\begin{eqnarray*}
 \overline{P_f}=\frac{1}{2}\sum_{n=1}^\infty |b_n|^2 = \frac{1}{2}\left(\frac{4}{\pi}\right)^2\left(1 + \frac{1}{3^2}+ \frac{1}{5^2}+ \frac{1}{7^2}+\cdots\right)
\end{eqnarray*}
usando a equação (\ref{serie_inv_impar}) da página \pageref{serie_inv_impar}, temos:
\begin{eqnarray*}
 \overline{P_f}=\frac{1}{2}\left(\frac{4}{\pi}\right)^2\frac{\pi^2}{8}=1
\end{eqnarray*}
\end{ex}

\section{Fenômeno de Gibbs}
A convergência das somas parciais da série de Fourier de uma função suave por partes em torno de um salto apresenta oscilações cujas amplitudes não convergem para zero. A convergência ponto a ponto acontece, mas se olharmos para o valor absoluto da diferença entre a função e soma parcial sempre encontramos um ponto onde esse valor é aproximadamente 8,9\% da amplitude do salto. Esse fenômeno é chamado de Fenômeno de Gibbs

\begin{figure}[!ht]
\begin{center}
\psset{unit =2cm, linewidth=1\pslinewidth}
 \begin{pspicture}(-1.3,-1.3)(4.3,1.3)
 \psaxes{->}(0,0)(-1.1,-1.2)(4.2,1.2)
\psplot[plotstyle=curve,linewidth=\pslinewidth,linecolor=black]{-1}{0}{-1}
\psplot[plotstyle=curve,linewidth=\pslinewidth,linecolor=black]{0}{1}{1}
\psplot[plotstyle=curve,linewidth=\pslinewidth,linecolor=black]{1}{2}{-1}
\psplot[plotstyle=curve,linewidth=\pslinewidth,linecolor=black]{2}{3}{1}
\psplot[plotstyle=curve,linewidth=\pslinewidth,linecolor=black]{3}{4}{-1}

%\psplot[plotstyle=curve,linewidth=\pslinewidth,linecolor=red,numpoints=400]{-1}{4}{x 180 mul sin 4 mul 3.1416 div }

%\psplot[plotstyle=curve,linewidth=\pslinewidth,linecolor=red,numpoints=400]{-1}{4}{x 180 mul sin x 180 mul 3 mul sin 9 div add 4 mul 3.1416 div }
%\psplot[plotstyle=curve,linewidth=\pslinewidth,linecolor=red,numpoints=400]{-1}{4}{x 180 mul sin x 180 mul 3 mul sin 3 div add x 180 mul 5 mul sin 5 div add 4 mul 3.1416 div }

\psplot[plotstyle=curve,linewidth=.4pt,linecolor=red,plotpoints=1000]{-1}{4}{x 180 mul sin x 180 mul 3 mul sin 3 div add x 180 mul 5 mul sin 5 div add x 180 mul 7 mul sin 7 div add x 180 mul 9 mul sin 9 div add 4 mul 3.1416 div }

\psplot[plotstyle=curve,linewidth=.4pt,linecolor=green,plotpoints=1000]{-1}{4}{x 180 mul sin x 180 mul 3 mul sin 3 div add x 180 mul 5 mul sin 5 div add x 180 mul 7 mul sin 7 div add x 180 mul 9 mul sin 9 div add x 180 mul 11 mul sin 11 div add x 180 mul 13 mul sin 13 div add x 180 mul 15 mul sin 15 div add x 180 mul 17 mul sin 17 div add 4 mul 3.1416 div }

\psplot[plotstyle=curve,linewidth=.4pt,linecolor=blue,plotpoints=4000]{-1}{4}{x 180 mul sin x 180 mul 3 mul sin 3 div add x 180 mul 5 mul sin 5 div add x 180 mul 7 mul sin 7 div add x 180 mul 9 mul sin 9 div add x 180 mul 11 mul sin 11 div add x 180 mul 13 mul sin 13 div add x 180 mul 15 mul sin 15 div add x 180 mul 17 mul sin 17 div add x 180 mul 19 mul sin 19 div add x 180 mul 21 mul sin 21 div add x 180 mul 23 mul sin 23 div add x 180 mul 25 mul sin 25 div add x 180 mul 27 mul sin 27 div add x 180 mul 29 mul sin 29 div add x 180 mul 31 mul sin 31 div add 4 mul 3.1416 div }


\rput(.3,1.3){$y=g(t)$}
\rput(4.1,.1){$t$}
\end{pspicture}
\end{center}
\end{figure}



\begin{figure}[!ht]
\begin{center}
\psset{yunit =2cm,xunit=20cm, linewidth=1\pslinewidth}
 \begin{pspicture}(-.1,.75)(0.5,1.25)
 \psaxes{->,Dx=.1}(0,0)(-.01,-.8)(0.5,1.2)

\psplot[plotstyle=curve,linewidth=.4pt,linecolor=red,plotpoints=1000]{0}{.5}{x 180 mul sin x 180 mul 3 mul sin 3 div add x 180 mul 5 mul sin 5 div add x 180 mul 7 mul sin 7 div add x 180 mul 9 mul sin 9 div add 4 mul 3.1416 div }

\psplot[plotstyle=curve,linewidth=.4pt,linecolor=green,plotpoints=1000]{0}{.5}{x 180 mul sin x 180 mul 3 mul sin 3 div add x 180 mul 5 mul sin 5 div add x 180 mul 7 mul sin 7 div add x 180 mul 9 mul sin 9 div add x 180 mul 11 mul sin 11 div add x 180 mul 13 mul sin 13 div add x 180 mul 15 mul sin 15 div add x 180 mul 17 mul sin 17 div add 4 mul 3.1416 div }

\psplot[plotstyle=curve,linewidth=.4pt,linecolor=blue,plotpoints=4000]{0}{.5}{x 180 mul sin x 180 mul 3 mul sin 3 div add x 180 mul 5 mul sin 5 div add x 180 mul 7 mul sin 7 div add x 180 mul 9 mul sin 9 div add x 180 mul 11 mul sin 11 div add x 180 mul 13 mul sin 13 div add x 180 mul 15 mul sin 15 div add x 180 mul 17 mul sin 17 div add x 180 mul 19 mul sin 19 div add x 180 mul 21 mul sin 21 div add x 180 mul 23 mul sin 23 div add x 180 mul 25 mul sin 25 div add x 180 mul 27 mul sin 27 div add x 180 mul 29 mul sin 29 div add x 180 mul 31 mul sin 31 div add 4 mul 3.1416 div }

\psplot[plotstyle=curve,linewidth=.4pt,linecolor=violet,plotpoints=12000]{0}{.5}{x 180 mul sin x 180 mul 3 mul sin 3 div add x 180 mul 5 mul sin 5 div add x 180 mul 7 mul sin 7 div add x 180 mul 9 mul sin 9 div add x 180 mul 11 mul sin 11 div add x 180 mul 13 mul sin 13 div add x 180 mul 15 mul sin 15 div add x 180 mul 17 mul sin 17 div add x 180 mul 19 mul sin 19 div add x 180 mul 21 mul sin 21 div add x 180 mul 23 mul sin 23 div add x 180 mul 25 mul sin 25 div add x 180 mul 27 mul sin 27 div add x 180 mul 29 mul sin 29 div add x 180 mul 31 mul sin 31 div add x 180 mul 33 mul sin 33 div add x 180 mul 35 mul sin 35 div add x 180 mul 37 mul sin 37 div add x 180 mul 39 mul sin 39 div add x 180 mul 41 mul sin 41 div add x 180 mul 43 mul sin 43 div add x 180 mul 45 mul sin 47 div add x 180 mul 41 mul sin 47 div add x 180 mul 49 mul sin 49 div add x 180 mul 51 mul sin 51 div add x 180 mul 53 mul sin 53 div add x 180 mul 55 mul sin 55 div add x 180 mul 57 mul sin 57 div add x 180 mul 59 mul sin 59 div add x 180 mul 61 mul sin 61 div add x 180 mul 63 mul sin 63 div add x 180 mul 65 mul sin 65 div add x 180 mul 67 mul sin 67 div add x 180 mul 69 mul sin 69 div add 4 mul 3.1416 div }


\end{pspicture}
\end{center}
\end{figure}

%\end{document}