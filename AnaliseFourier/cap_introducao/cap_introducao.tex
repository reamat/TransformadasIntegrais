%Este trabalho está licenciado sob a Licença Creative Commons Atribuição-CompartilhaIgual 3.0 Não Adaptada. Para ver uma cópia desta licença, visite https://creativecommons.org/licenses/by-sa/3.0/ ou envie uma carta para Creative Commons, PO Box 1866, Mountain View, CA 94042, USA.

\chapter{Introdução}

Já pensou em como é poderosa a ferramenta da representação?
Transmissão de calor, música, sinais elétricos, posicionamento terrestre por satélite, criptografia, análise combinatória, televisão, rádio, monitoramento do nível do mar. O que todas essas coisas têm em comum? Todas elas podem ser representadas por um conceito que vamos introduzir neste capítulo: Séries e Transformadas de Fourier. Elas permitem analisar fenômenos periódicos e não periódicos, o que possibilita uma vasta gama de aplicações e interpretações de grandezas da natureza.
    No século XVIII, auge do Iluminismo, a busca pelas leis que regem a natureza teve um de seus momentos mais intensos. Houve uma série de avanços em todos os ramos da ciência. Porém, na Mecânica houve uma especial intensidade: era o momento em que máquinas térmicas estavam sendo apresentadas e termômetros aperfeiçoados. Uma das grandes discussões da época girava em torno do como determinar a propagação de calor em corpos sólidos.
    É neste contexto que Jean Joseph Baptiste Fourier (1768-1830) inicia seus estudos sobre a transmissão de calor, com o grande diferencial de que sua análise ignorava a natureza do calor, ou seja, observou exclusivamente a propagação. A equação da condução do calor foi obtida por meio de equações diferenciais parciais e a solução foi desenvolvida através de séries trigonométricas.
A apresentação de séries como solução para problemas de contorno foi uma grande quebra de paradigma na época. Pois existia o consenso em caracterizar uma função f(x) se, e somente se, f(x) pudesse ser representada por uma expressão bem comportada, ou seja, sem descontinuidade, vértices, lacunas, cúspides, diferenciável em todo intervalo, etc. Porém Fourier afirmava que gráficos com descontinuidades podia ser representado através de séries trigonométricas, e portanto deviam ser considerados funções de fato.
Após a morte de Fourier, Dirichlet enunciou um teorema que apresenta as condições suficientes para uma função possuir representação em Série de Fourier. Tal teorema  ficou conhecido como as condições de Dirichlet para que uma função periódica possa ter uma representação em formato de séries de Fourier.  O grande diferencial das séries de Fourier, comparado à séries de potências, em dadas circunstâncias, é que a primeira tem sua convergência válida para todo o domínio da função.
Atualmente, a Análise de Fourier possui aplicações em diversas áreas do conhecimento (teoria de comunicação, sinais e sistemas por exemplo).

\emconstrucao
