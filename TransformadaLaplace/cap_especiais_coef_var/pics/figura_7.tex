\documentclass{standalone}\usepackage{pstricks-add}\usepackage{pstricks,pst-plot}\usepackage[dvips]{graphicx}\usepackage{pst-math}\usepackage{pst-plot}\usepackage{pst-circ}\usepackage[brazil]{babel}\usepackage[utf8]{inputenc}\usepackage[T1]{fontenc}\usepackage{amsmath}\usepackage{amssymb}\usepackage{amsthm}\usepackage{mathtools}\newcommand{\sen}{\operatorname{sen}\,}\newcommand{\senh}{\operatorname{senh}\,}\renewcommand{\sin}{\operatorname{sen}\,}\renewcommand{\sinh}{\operatorname{senh}\,}\begin{document}\psset{xunit =1.0cm,yunit=1.0cm, linewidth=1\pslinewidth}
 \begin{pspicture}(-0.3,-1.0)(10.0,1.5)
 \psaxes{->}(0,0)(0.0,-0.2)(9.5,1.2)
\psplot[linecolor=blue,plotstyle=curve,plotpoints=200]{0.0}{1.0}{0}
\psplot[linecolor=blue,plotstyle=curve,plotpoints=200]{1.0}{2.0}{x 1 sub }
\psplot[linecolor=blue,plotstyle=curve,plotpoints=200]{2.0}{3.0}{3 x sub }
\psplot[linecolor=blue,plotstyle=curve,plotpoints=200]{3.0}{4.0}{0}
\psplot[linecolor=blue,plotstyle=curve,plotpoints=200]{4.0}{5.0}{x 4 sub }
\psplot[linecolor=blue,plotstyle=curve,plotpoints=200]{5.0}{6.0}{6 x sub }
\psplot[linecolor=blue,plotstyle=curve,plotpoints=200]{6.0}{7.0}{0}
\psplot[linecolor=blue,plotstyle=curve,plotpoints=200]{7.0}{8.0}{x 7 sub }
\psplot[linecolor=blue,plotstyle=curve,plotpoints=200]{8.0}{9.0}{9 x sub }
\rput(9.6,.3){$t$}
\rput(0.0,1.5){$f(t)$}
\end{pspicture}
\end{document}