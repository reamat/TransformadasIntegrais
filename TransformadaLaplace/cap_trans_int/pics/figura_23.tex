\documentclass{standalone}\usepackage{pstricks-add}\usepackage{pstricks,pst-plot}\usepackage[dvips]{graphicx}\usepackage{pst-math}\usepackage{pst-plot}\usepackage{pst-circ}\usepackage[brazil]{babel}\usepackage[utf8]{inputenc}\usepackage[T1]{fontenc}\usepackage{amsmath}\usepackage{amssymb}\usepackage{amsthm}\usepackage{mathtools}\newcommand{\sen}{\operatorname{sen}\,}\newcommand{\senh}{\operatorname{senh}\,}\renewcommand{\sin}{\operatorname{sen}\,}\renewcommand{\sinh}{\operatorname{senh}\,}\begin{document}\psset{xunit =1cm,yunit=1cm, linewidth=1\pslinewidth}
 \begin{pspicture}(-0.6,-0.8)(6.5,5.0)
 \psaxes[labels]{->}(0,0)(-0.5,-0.2)(6.5,4.5)
\psplot[linecolor=blue,plotstyle=curve,plotpoints=200]{-0.5}{0.0}{0}
\psline[linecolor=blue](0.0,0)(0,1)
\psplot[linecolor=blue,plotstyle=curve,plotpoints=200]{0.0}{2.0}{1}
\psline[linecolor=blue](2.0,1)(2.0,2)
\psplot[linecolor=blue,plotstyle=curve,plotpoints=200]{2.0}{4.0}{2}
\psline[linecolor=blue](4.0,2)(4.0,3)
\psplot[linecolor=blue,plotstyle=curve,plotpoints=200]{4.0}{6.0}{3}
\psline[linecolor=blue](6.0,3)(6.0,4)
\psplot[linecolor=blue,plotstyle=curve,plotpoints=200]{6.0}{6.5}{4}
\rput(6.6,.3){$t$}
\rput(0.0,4.7){$f(t)$}
\end{pspicture}
\end{document}
