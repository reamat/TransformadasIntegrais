\documentclass[10pt,a4paper]{article}%
\usepackage{amsmath,amsfonts,amstext,amsthm}%
\usepackage{pstricks,pst-plot}%
\usepackage[dvips]{graphicx}%
\usepackage[brazil]{babel}%
\usepackage[latin1]{inputenc}%

\pagestyle{empty}%

\setlength{\topmargin}{-2.5cm} \setlength{\oddsidemargin}{-1cm} %%% top margin -3
\setlength{\evensidemargin}{1cm} \setlength{\textheight}{27cm}
\setlength{\textwidth}{18cm}


\renewcommand{\sin}{\operatorname{sen}}
\renewcommand{\sinh}{\operatorname{senh}}

\begin{document}

{\bf Exemplo 11} 
{\bf Item a}
\begin{eqnarray*}
F(s)&=&\frac{s^2-6s+4}{s^3-3s^2+2s}=\frac{s^2-6s+4}{s(s^2-3s+2)}=\frac{s^2-6s+4}{s(s-1)(s-2)}=\frac{A}{s}+\frac{B}{s-1}+\frac{C}{s-2}
\end{eqnarray*}

\begin{eqnarray*}
A&=&\lim_{s\to 0}sF(s)=\lim_{s\to 0}\frac{s^2-6s+4}{(s-1)(s-2)}=\left.\frac{s^2-6s+4}{(s-1)(s-2)}\right|_{s=0}=2\\
B&=&\lim_{s\to 1}(s-1)F(s)=\lim_{s\to 1}\frac{s^2-6s+4}{s(s-2)}=\left.\frac{s^2-6s+4}{s(s-2)}\right|_{s=1}=1\\
C&=&\lim_{s\to 2}(s-2)F(s)=\lim_{s\to 2}\frac{s^2-6s+4}{s(s-1)}=\left.\frac{s^2-6s+4}{s(s-1)}\right|_{s=2}=-2\\
\end{eqnarray*}

\begin{eqnarray*}
\mathcal{L}^{-1}\left\{F(s)\right\}&=&\mathcal{L}^{-1}\left\{\frac{2}{s}+\frac{1}{s-1}-\frac{2}{s-2}\right\}\\
&=&2+e^{t}-2e^{2t},t>0
\end{eqnarray*}



{\bf Item b}
\begin{eqnarray*}
F(s)&=&\frac{s^2+s-2}{(s+1)^3}=\frac{A}{(s+1)^3}+\frac{B}{(s+1)^2}+\frac{C}{(s+1)}
\end{eqnarray*}

\begin{eqnarray*}
A&=&\lim_{s\to -1}(s+1)^3F(s)=\lim_{s\to 1}(s^2+s-2)=\left.(s^2+s-2)\right|_{s=-1}=-2\\
\end{eqnarray*}
Conhecendo o valor de $A$, reduzimos o problema a
\begin{eqnarray*}
\frac{s^2+s-2}{(s+1)^3}+\frac{2}{(s+1)^3}=\frac{B}{(s+1)^2}+\frac{C}{(s+1)}
\end{eqnarray*}
Simplificando o lado da esquerda, temos:
\begin{eqnarray*}
\frac{s^2+s-2}{(s+1)^3}+\frac{2}{(s+1)^3}=\frac{s^2+s}{(s+1)^3}=\frac{s(s+1)}{(s+1)^3}=\frac{s}{(s+1)^2}
\end{eqnarray*}
\begin{eqnarray*}
B&=&\lim_{s\to -1}(s+1)^2\frac{s}{(s+1)^2}=-1\\
\end{eqnarray*}
Reduzimos mais uma vez o problema a
\begin{eqnarray*}
\frac{s}{(s+1)^2}+\frac{1}{(s+1)^2}=\frac{C}{s+1}
\end{eqnarray*}
Simplificando o lado da esquerda, temos:
\begin{eqnarray*}
\frac{s}{(s+1)^2}+\frac{1}{(s+1)^2}=\frac{s+1}{(s+1)^2}=\frac{1}{s+1}
\end{eqnarray*}
o que implica $C=1$. Portanto temos:

\begin{eqnarray*}
F(s)&=&\frac{s^2+s-2}{(s+1)^3}=-\frac{2}{(s+1)^3}-\frac{1}{(s+1)^2}+\frac{1}{(s+1)}
\end{eqnarray*}
e, assim,
\begin{eqnarray*}
f(t)&=&e^{-t}\left(-t^2-t+1\right)
\end{eqnarray*}



{\bf Item c}
\begin{eqnarray*}
F(s)&=&\frac{s}{(s+1)^3}=\frac{s+1}{(s+1)^3}-\frac{1}{(s+1)^3}=\frac{1}{(s+1)^2}-\frac{1}{(s+1)^3}
\end{eqnarray*}
\begin{eqnarray*}
f(t)&=&e^{-t}\left(t-\frac{t^2}{2}\right)
\end{eqnarray*}


{\bf item d}
\begin{eqnarray*}
F(s)&=&\frac{5s^2-15s-11}{(s+1)(s-2)^3}=\frac{A}{s+1}+\frac{B}{(s-2)^3}+\frac{C}{(s-2)^2}+\frac{D}{(s-2)}
\end{eqnarray*}

\begin{eqnarray*}
A&=&\lim_{s\to -1}(s+1)F(s)=-\frac{1}{3}\\
B&=&\lim_{s\to 2}(s-2)^3F(s)=-7\\
\end{eqnarray*}
Obt�m-se tamb�m
\begin{eqnarray*}
C&=&4\\
D&=&\frac{1}{3}
\end{eqnarray*}
Portanto
$$f(t)=-\frac{1}{3}e^{-t}+\left(-\frac{7}{2}t^2+4t+\frac{1}{3}\right)e^{2t}$$

{\bf Item e}
\begin{eqnarray*}
F(s)&=&\frac{3s^2-2s-1}{(s-3)(s^2+1)}=\frac{A}{s-3}+\frac{Bs+C}{s^2+1}
\end{eqnarray*}

\begin{eqnarray*}
A=\lim_{s\to 3}(s-3)F(s)=2\\
\end{eqnarray*}

\begin{eqnarray*}
Bs+C&=&\left(F(s)-\frac{A}{s-3}\right)(s^2+1)\\
&=&\left(\frac{3s^2-2s-1}{(s-3)(s^2+1)}-\frac{2}{s-3}\right)(s^2+1)\\
&=&\left(\frac{3s^2-2s-1-2(s^2+1)}{(s-3)(s^2+1)}\right)(s^2+1)\\
&=&\frac{s^2-2s-3}{(s-3)}=\frac{(s+1)(s-3)}{(s-3)}=s+1
\end{eqnarray*}
Assim, $B=C=1$ e temos
$f(t)=e^{3t}+\cos(t)+\sin(t)$

{\bf Item f}
\begin{eqnarray*}
F(s)&=&\frac{s^2+2s+3}{(s^2+2s+2)(s^2+2s+5)}=\frac{As+B}{s^2+2s+2}+\frac{Cs+D}{s^2+2s+5}\\
&=&\frac{(As+B)(s^2+2s+5)+(Cs+D)(s^2+2s+2)}{(s^2+2s+2)(s^2+2s+5)}\\
&=&\frac{(A+C)s^3+(2A+2C+B+D)s^2+(5A+2D+2B+2C)s+(5B+2D)}{(s^2+2s+2)(s^2+2s+5)}\\
\end{eqnarray*}
Comparando coeficientes, temos:
\begin{eqnarray*}
A+C&=&0\\
2A+2C+B+D&=&1\\
5A+2D+2B+2C&=&2\\
5B+2D&=&3
\end{eqnarray*}
\begin{eqnarray*}
A&=&0  \\
B&=&\frac{1}{3}  \\
C&=& 0 \\
D&=&\frac{2}{3}  
\end{eqnarray*}

\begin{eqnarray*}
F(s)&=&\frac{1}{3}\frac{1}{s^2+2s+2}+\frac{2}{3}\frac{1}{s^2+2s+5}\\
&=&\frac{1}{3}\frac{1}{(s+1)^2+1}+\frac{2}{3}\frac{1}{(s+1)^2+2^2}
\end{eqnarray*}
\begin{eqnarray*}
f(t)=\frac{1}{3}e^{-t}\left[\sin(t)+\sin(2t)\right]
\end{eqnarray*}

{\bf Item g}
\begin{eqnarray*}
F(s)&=&\frac{1}{s^2(s-2)}=\frac{A}{s}+\frac{B}{s^2}+\frac{C}{s-2}
\end{eqnarray*}


\newpage


{\bf Quest�o 2 - Lista 6}

Para resolver esta quest�o, usaremos a seguinte identidade envolvendo a fun��o Gama de Euler:

\begin{eqnarray*}
\Gamma\left(\frac{2n+1}{2}\right)&=&\frac{(2n)!}{n!}\frac{\sqrt{\pi}}{2^{2n}}=\frac{(2n-1)!}{(n-1)!}\frac{\sqrt{\pi}}{2^{2n-1}}
\end{eqnarray*}
Para provar esta identidade, usamos a rela��o de recorr�ncia
$$\Gamma(x+1)=x\Gamma(x)$$
e o fato bem conhecido  de que
$$\Gamma(1/2)=\sqrt{\pi}$$
Para obter
\begin{eqnarray*}
\Gamma(3/2)&=&\Gamma(1+1/2)=\frac{1}{2}\Gamma(1/2)=\frac{\sqrt{\pi}}{2}\\
\Gamma(5/2)&=&\Gamma(1+3/2)=\frac{3}{2}\Gamma(3/2)=\frac{3\sqrt{\pi}}{2^2}\\
\Gamma(7/2)&=&\Gamma(1+5/2)=\frac{5}{2}\Gamma(5/2)=\frac{5 \cdot 3\sqrt{\pi}}{2^3}\\
\Gamma(9/2)&=&\Gamma(1+7/2)=\frac{7}{2}\Gamma(7/2)=\frac{7\cdot 5 \cdot 3\sqrt{\pi}}{2^4}\\
\end{eqnarray*}
E generalizando, temos:
\begin{eqnarray*}
\Gamma\left(\frac{2n+1}{2}\right)&=&\Gamma\left(1+\frac{2n-1}{2}\right)=\frac{2n-1}{2}\Gamma\left(\frac{2n-1}{2}\right)={(2n-1)(2n-3)\cdots 3}\frac{\sqrt{\pi}}{2^{n}}\\
&=&\frac{(2n-1)(2n-2)(2n-3)\cdots 3\cdot 2 \cdot 1}{(2n-2)(2n-4)\cdots 2}\frac{\sqrt{\pi}}{2^{n}}\\
&=&\frac{(2n-1)(2n-2)(2n-3)\cdots 3\cdot 2 \cdot 1}{2^{n-1}(n-1)(n-2)\cdots 1}\frac{\sqrt{\pi}}{2^{n}}\\
&=&\frac{(2n-1)!}{(n-1)!}\frac{\sqrt{\pi}}{2^{2n-1}}
\end{eqnarray*}

Retornando ao problema principal, calculamos:

\begin{eqnarray*}
f(t)&=&\sin\sqrt{t}= \sum_{n=0}^{\infty}(-1)^n\frac{t^{(2n+1)/2}}{(2n+1)!}\\
F(s)&=&\mathcal{L}\left\{f(t)\right\}=\sum_{n=0}^{\infty}\frac{(-1)^n}{{(2n+1)!}}\mathcal{L}\left\{{t^{(2n+1)/2}}\right\}\\
&=&\sum_{n=0}^{\infty}\frac{(-1)^n}{{(2n+1)!}}\mathcal{L}\left\{{t^{(2n+1)/2}}\right\}\\
&=&\sum_{n=0}^{\infty}\frac{(-1)^n}{{(2n+1)!}}\Gamma\left((2n+3)/2\right)\frac{1}{s^{(2n+3)/2}}\\
&=&\sum_{n=0}^{\infty}\frac{(-1)^n}{{(2n+1)!}}\frac{(2n+1)!}{n!}\frac{\sqrt{\pi}}{2^{2n+1}}\frac{1}{s^{(2n+3)/2}}\\
&=&\frac{\sqrt{\pi}}{2s^{3/2}}\sum_{n=0}^{\infty}{(-1)^n}\frac{1}{n!}\frac{1}{2^{2n}}\frac{1}{s^{n}}\\
&=&\frac{\sqrt{\pi}}{2s^{3/2}}\sum_{n=0}^{\infty}\frac{1}{n!}{\left(-\frac{1}{4s}\right)^{n}}\\
&=&\frac{\sqrt{\pi}}{2s^{3/2}}e^{-\frac{1}{4s}}
\end{eqnarray*}

\end{document}